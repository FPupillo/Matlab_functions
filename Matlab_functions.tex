% This LaTeX was auto-generated from MATLAB code.
% To make changes, update the MATLAB code and export to LaTeX again.

\documentclass{article}

\usepackage[utf8]{inputenc}
\usepackage[T1]{fontenc}
\usepackage{lmodern}
\usepackage{graphicx}
\usepackage{color}
\usepackage{hyperref}
\usepackage{amsmath}
\usepackage{amsfonts}
\usepackage{epstopdf}
\usepackage[table]{xcolor}
\usepackage{matlab}

\sloppy
\epstopdfsetup{outdir=./}
\graphicspath{ {./Matlab_functions_images/} }

\begin{document}

\begin{par}
\begin{flushleft}
Matlab functions
\end{flushleft}
\end{par}

\begin{matlabcode}
% concatenate strings and numbers
a = 2;
newVar = ['a' num2str(a)];
newVar
\end{matlabcode}
\begin{matlaboutput}
newVar = 'a2'
\end{matlaboutput}


\begin{par}
\begin{flushleft}
Get a variable from a string
\end{flushleft}
\end{par}

\begin{matlabcode}
a2 = 100;
\end{matlabcode}
\begin{matlaboutput}
a2 = 100
\end{matlaboutput}
\begin{matlabcode}
b = eval('a2');
b
\end{matlabcode}
\begin{matlaboutput}
b = 100
\end{matlaboutput}


\begin{par}
\begin{flushleft}
turn a cell array into a string array
\end{flushleft}
\end{par}

\begin{matlabcode}
% Create a cell array
myCell = {'a';'b';'c'}
\end{matlabcode}
\begin{matlaboutput}
myCell = 3x1 cell    
'a'         
'b'         
'c'         

\end{matlaboutput}
\begin{matlabcode}

% turn it into a string
myCell_str= string(myCell)
\end{matlabcode}
\begin{matlaboutput}
myCell_str = 3x1 string    
"a"         
"b"         
"c"         

\end{matlaboutput}
\begin{matlabcode}

% concatenate the three element in one single element
myCell_str_sing = strjoin(myCell_str, '_')
\end{matlabcode}
\begin{matlaboutput}
myCell_str_sing = "a b c"
\end{matlaboutput}
\begin{matlabcode}

% this is a string array (double quotes)
% to convert it to a character vector run the following
myCell_char = char(myCell_str_sing)

\end{matlabcode}

\end{document}
